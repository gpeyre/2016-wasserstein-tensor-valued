% !TEX root = ../TensorOT.tex


%%%%%%%%%%%%%%%%%%%%%%%%%%%%%%%%%%%%%%%%%%%%%%%%%%%%%%%%%
\subsection{Diffusion Tensor Imaging}

Diffusion tensor magnetic resonance imaging (DTI) is a popular technic to image the white matter of the brain (see~\cite{wandell2016clarifying} for a recent overview). DTI measures the diffusion of water molecules, which can be compactly encoded using a PSD tensor field $\mu(x) \in \Ss_+^3$, whose anisotropy and size matches the local diffusivity. 
%
A typical goal of this imaging technic is to map the brain anatomical connectivity, and in particular track the  white matter fibers. This require a careful handling of the tensor's mass (trace) and anisotropy, so that using Q-OT is a perfect fit for such data.

Data from~\cite{CaiafaPestilli}.