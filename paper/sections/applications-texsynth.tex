% !TEX root = ../TensorOT.tex


%%%%%%%%%%%%%%%%%%%%%%%%%%%%%%%%%%%%%%%%%%%%%%%%%%%%%%%%%
\subsection{Spectral Color Texture Synthesis}

As advocated initially in~\cite{galerne2011random}, a specific class of textured images (so-called micro-textures) is well-modeled using stationary Gaussian fields. In the following, we denote $p$ the pixel positions and $x$ the Fourier frequency indices. For color images, these fields are fully characterized by their mean $m \in \RR^3$ and their Fourier power spectrum, which is a tensor valued field $\mu(x)$ where, for each frequency $x$ (defined on a 2-D grid) $\mu(x) \in \CC^{3 \times 3}$ is a complex positive semi-definite hermitian matrix. 


In practice, $\mu(x)$ is estimated from an exemplar color image $f(p) \in \RR^3$ using an empirical spectrogram 
\eql{\label{eq-power-spectrum}
	\mu(x) \eqdef \frac{1}{K} \sum_{k=1}^K \hat f_k(x) \hat f_k(x)^* \in \CC^{3 \times 3}
}
where $\hat f_k$ is the Fourier transform of $f_k(p) \eqdef f(p) w_k(p)$ (computed using the FFT), $w_k$ are windowing functions centred around $K$ locations in the image plane, and $v^* \in \CC^{1 \times 3}$ denoted the transpose-conjugate of a vector $v \in \CC^{3 \times 1}$. 
%
Increasing the number $K$ of windowed estimations helps to avoid having rank-deficient covariances ($K=1$ leads to a field $\mu$ of rank-1 tensors).

Randomized new textures are then created by generating random samples $F(p) \in \RR^3$ from the Gaussian field, which is achieved by defining the Fourier transform $\hat F(x) \eqdef m + \hat N(x) \sqrt{\mu(x)} \ones_3$, where $N(p)$ is the realization of a Gaussian white noise, and $\sqrt{\cdot}$ is the matrix square root (see~\cite{galerne2011random} for more details).

Figure~\ref{fig:texsynth} shows an application where two input power spectra $(\mu,\nu)$ (computed using~\eqref{eq-power-spectrum} from two input textures exemplars $(f,g)$)  are interpolated using~\eqref{eq-interpolating}, and for each interpolation parameter $t \in [0,1]$ a new texture $F$ is synthesized and displayed.
%
Note that while the Q-Sinkhorn Algorithm~\ref{alg:sinkhorn} is provided for real PSD matrices, it extends verbatim to complex positive hermitian matrices (the matrix logarithm and exponential being defined the same way as for real matrices).


