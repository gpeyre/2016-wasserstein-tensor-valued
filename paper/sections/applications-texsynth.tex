% !TEX root = ../TensorOT.tex


%%%%%%%%%%%%%%%%%%%%%%%%%%%%%%%%%%%%%%%%%%%%%%%%%%%%%%%%%
\subsection{Spectral Color Texture Synthesis}

As advocated initially in~\cite{galerne2011random}, a specific class of textured images (so-called micro-textures) is well modelled using stationary Gaussian fields. For color images, these fields are fully characterised by their mean $m \in \RR^3$ and their Fourier power spectrum, which is a tensor valued field $\mu(x)$ where, for each frequency $x$ (defined on a 2-D grid) $\mu(x) \in \CC^{3 \times 3}$ is a complex positive semi-definite hermitian matrix. 
%
In practice, $\mu(x)$ is estimated from an exemplar color image $f(p) \in \RR^3$ using an empirical spectrogram 
\eql{\label{eq-power-spectrum}
	\mu(x) = \frac{1}{K} \sum_{k=1}^K \Ff(w_k \cdot f)  \Ff(w_k \cdot f)^*  \in \CC^{3 \times 3}
	% \sum_{p} w_k(p) f(p) e^{\imath \dotp{p}{x}}
}
where $\Ff$ is the discrete Fourier transform (computed using the FFT)
and where $w_k$ are windowing functions centred around $K$ locations in the image plane. 
%
Increasing the number $K$ of windowed estimation helps to avoid having rank-deficient covariances ($K=1$ leads to a field $\mu$ of rank-1 tensors).
%
Randomized new textures are then generated by generating random samples $F(p) \in \RR^3$ from the Gaussian field, which is achieved using the formula $F = \Ff^{-1}( \hat F)$ where the Fourier transform reads $\hat F(x) = N(x) \sqrt{\mu(x)} \ones_3$, where $N(x)$ is the realization of a Gaussian white noise, and $\sqrt{\cdot}$ is the matrix square root.
%
Figure~\ref{fig:texsynth} shows an application where two input power spectra $(\mu,\nu)$ (computed using~\eqref{eq-power-spectrum} from two input textures exemplar $(f_0,f_1)$)  are interpolated using~\eqref{eq-interpolating}, and for each interpolation parameter $t \in [0,1]$ a new texture $F$ is synthesized and displayed.
%
Note that while the Q-Sinkhorn Algorithm~\ref{alg:sinkhorn} is detailed for real PSD matrices, it extends verbatim to complex positive hermitian matrices (the matrix logarithm and exponential being defined the same way as for real matrices).



%%% FIG %%%
\newcommand{\TexSynthImg}[1]{\includegraphics[width=.195\linewidth]{texsynth/#1}}
\begin{figure}\centering
\begin{tabular}{@{}c@{\hspace{1mm}}c@{\hspace{1mm}}c@{\hspace{1mm}}c@{\hspace{1mm}}c@{}}
\TexSynthImg{spectrum-1}&
\TexSynthImg{spectrum-3}&
\TexSynthImg{spectrum-5}&
\TexSynthImg{spectrum-7}&
\TexSynthImg{spectrum-9}\\
\TexSynthImg{synthesis-1}&
\TexSynthImg{synthesis-3}&
\TexSynthImg{synthesis-5}&
\TexSynthImg{synthesis-7}&
\TexSynthImg{synthesis-9}\\
$t=0$ & $t=1/4$ & $t=1/2$ & $t=3/4$ & $t=1$
\end{tabular}
\begin{tabular}{@{}c@{\hspace{5mm}}c@{}}
\TexSynthImg{original-1}&
\TexSynthImg{original-2}\\
$f_0$  & $f_1$
\end{tabular}
\caption{\textbf{First row:}  display $\tr(\mu_t(x))$ where $\mu_t$ are the interpolated power spectra. 
\textbf{Second row:} realizations of the Gaussian field parameterized by the power spectra  $\mu_t$. 
\textbf{Third row:} input texture exemplars from which $(\mu_{t=0},\mu_{t=1})=(\mu,\nu)$ are computed.
} \label{fig:texsynth}
\end{figure}
%%% FIG %%%