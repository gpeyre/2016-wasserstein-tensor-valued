

%%%%%%%%%%%%%%%%%%%%%%%%%%%%%%%%%%%%%%%%%%%%
\subsection{Shape Registration}

Only use $W$ as a fidelity term in diffeomorphic registration when representing shapes as tensor measures. 

We denote $\mu$ the input measure to register to a target measure $\nu$, which are defined on $\RR^d$ and takes values in $\Ss_d^+$. 
%
The action of a diffeomorphism $T : \RR^d \rightarrow \RR^d$ on $\mu$ is defined as
\eq{
	T_\sharp \mu \eqdef \sum_i [T'(x_i)]^{-1} \mu_i [T'(x_i)]^{-1,\top} \de_{T(x_i)}.
}
This action of warpings on tensor-valued measures is more involved than for the case of scalar valued measures (as done for instance in~\cite{}). Indeed, this action not only displaces the input masses but also rotate and rescale the tensors by the conjugation by the Jacobian $T(x_i)^{-1} \in \RR^{d \times d}$. \gabriel{Show an example} 

Assuming a parametrization $\th \mapsto T_\th$ of the deformation (one could of course use more involved non-parametric model, for instance~\cite{daemon,lddmm}), the registration problem corresponds to solving 
\eq{
	\min_\th \Ee(\th) = W(T_{\th,\sharp} \mu, \nu), 
}
where the position $x(\th)$ are typically parameterized using a diffeomorphic model. 

Using the chain-rule, \gabriel{this formula is actually wrong because it lacks the differential of the conjugation term in $T_{\th,\sharp}$, which seems intractable to compute.}
\eq{
	\nabla \Ee(\th) = [\partial_\th T_\th(x)]^* [\nabla_x W(T_{\th,\sharp} \mu, \nu)], 
}
where $[\partial_\th T_\th(x)]^*$ is the adjoint of the differential of $\th \mapsto (T_\th(x_i))_i$.
%where typically the adjoint Jacobian $[\partial T(\th)]^*$ is implemented using a adjoint state method or reverse automatic�code differentiation.
%
Assuming for simplicity $c_{i,j} = \norm{x_i-y_j}^2\Id_{d \times d}$, the gradient of $x \mapsto  W(\mu,\nu)$ (where $\mu$ is parameterized by $x=(x_i)_i$) is given by 
\eq{
	\nabla_x W(\mu,\nu) = (2 \sum_j \tr(\ga_{i,j}) (x_i-y_j))_i
}
where $\ga$ is a solution of~\eqref{eq-kantorovitch-regul} computed using Sinkhorn iterations.