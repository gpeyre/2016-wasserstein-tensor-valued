%%%%%%%%%%%%%%%%%%%%%%%%%%%%%%%%%%%%%%%%%%%%%%%%%%%%%%%%%
\subsection{Deformation Interpolation by Stress Transportation}

We propose here to navigate between deformations of a given solid by transporting the stress tensors field generated by these deformations. The stress tensor encodes the local amount and anisotropy of  stretch of the deformation. The proposed interpolation method is able to transport inside the shape regions of high compression/expansion.
%
While state of the art interpolation method use involved non-convex variational methods (typically modelling deformation as a manifold of diffeomorphisms\gabriel{cite LDDMM}), the proposed method is more ad-hoc but rely on a convex solver. 


From some spatial deformation $x \in \Om \rightarrow T(x)$ defined on some domain $\Om \subset \RR^d$, one considers the Jacobian $J(x) \eqdef \partial_x T(x)$
and performs its polar decomposition $J(x) = U(x)�\mu(x)$ where $U(x) \in \Oo_d$ is orthogonal and $\mu(x) \in \Ss_d^+$ is PSD. 
%
Here $\mu$ is the the so-called stress tensor field \gabriel{is it the correct wording from  continuum mechanics ?}, and it describes the amount, direction and anisotropy of stretch that a deformable body $\Om$ is subject to through the deformation $T$.
%
Note that it is important to to polar decompose $J(x)$ rather than $J(x)^\top$ because this way all the PSD matrices $\mu(x)$ are defined in a common reference frame.


The deformation $T$ is assumed to be a smooth diffeomorphism (so that the shape is not self-interpenetrating) and supposed without loss of generality to be orientation preserving. Then $\det(U(x))=1$ so that there is no ``singularity" in this decomposition, $U$ is a smooth ``correction" field of rotations \gabriel{Is it true?}.

\begin{rem}[Smaller deformations]
If $T$ is not a diffeomorphism (for instance if it arises from some optical flow computation), then one can consider a ``smaller'' deformation $T^\de(x) \eqdef (1-\de) x +  \de (T(x)-x)$  so that for $\de$ small enough $T^\de$ becomes a diffeomorphism.
\end{rem}


\if 0
\begin{rem}[Small deformation asymptotic]
Note that as $\de \rightarrow 0$, one obtains the linearized decomposition of $J^\de(x) \eqdef \partial_x T^\de(x)$ as  $T^\de(x) = U^\de(x) \mu^\de(x)$ using
\begin{align*}
  U^\de(x) &= \Id_{d \times d} + \de D(x) + O(\de^2), \\
  \mu_t(x) &= \Id_{d \times d} + \de C(x) + O(\de^2)
\end{align*}
where $C(x) \eqdef (J(x)+J(x)^\top)/2-\Id_{d \times d}$ is the Cauchy stress tensor and $D(x) = (J(x)-J(x)^\top)/2$ is a deviator tensor anti-symetric part. This shows that typically $\de$ should be chosen of the order of $-1/\min_x(\text{eig}(C(x)))$ to obtain a proper diffeomorphism.
\end{rem}

\begin{rem}[Anisotropy boosting]
This parameter $\de$ controls the anisotropy of $\mu^\de$. This means that even if the input $T$ is a diffomorphism, one can ``boost" it to increase the anisotropy of $\mu$ by choosing $\de>1$. This is an extra parameter of the method to rescale the tensors.
\end{rem}
\fi

\gabriel{Maybe use $(T_0,T_1)$ in place of $(T,S)$. }
Given two input deformation field $T(x)$ and $S(x)$, we denote their Jacobian's decomposition $(J(x)=U(x) \mu(x),K(x)=V(x)\nu(x))$. We use quantum OT to interpolate between these deformation by interpolating the Jacobians. 

\gabriel{Explain sampling and discretization. }
We first solve the OT~\eqref{eq-Kantorovich} to obtain the coupling $\ga$ (using Sinkhorn's iterations) between $\mu$ and $\nu$. We then transport the rotation fields $(U,V)$ fields along the coupling. We simply update the formula~\eqref{eq-interpolating} to account also for the rotational part, and define
\eql{\label{eq-interpolating}
	\foralls t \in [0,1], \quad
	J_t \eqdef \sum_{i,j} \ga_{i,j}^t \de_{x_{i,j}^t}.
}
where now we define 
\eq{
	\ga_{i,j}^t \eqdef [(1-t) \bar\muA_i + t \bar\muB_{j}] \ga_{i,j}.
}
\eq{
	\bar\muA_i = J_i \Big( \sum_{j} \ga_{i,j} \Big)^{-1} 
	\qandq
	\bar\muB_j = K_j \Big( \sum_{i} \ga_{i,j} \Big)^{-1}
}
One reconstructs using a least square the deformation
\eq{
	T_t \eqdef \uargmin{T} \norm{\partial_x T-J_t}.
}
Note that actually this minimisation needs to be corrected to account for the kernel of $\partial_x$ \gabriel{I guess fixing the mean or a point is enough}. 
% where $\tilde T_t = (1-t) T+t S$ is for the linear interpolation (but other reference can be used).

This interpolation satisfies $T_{t=0}=T$ and $T_{t=1}=S$, and it should take into account compressions/dilatations/stretches occurring are different places in the shape. 

