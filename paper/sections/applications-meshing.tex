% !TEX root = ../TensorOT.tex


%%%%%%%%%%%%%%%%%%%%%%%%%%%%%%%%%%%%%%%%%%%%%%%%%%%%%%%%%
\subsection{Anisotropic Meshing}

Approximation with anisotropic piecewise linear finite elements on a triangulated mesh is a fundamental tool to address tasks such as discretizing partial differential equations, performing surface remeshing~\cite{alliez2003anisotropic} or for image compression~\cite{demaret2006image}.
%
A common practice is to generate triangulations complying with a PSD tensor field $\mu$, i.e. such that a triangle centred at $x$ should be inscribed in the ellipsoid $\enscond{u}{(u-x)^\top \mu(x) (u-x) \leq \de }$ for some $\de$ controlling the triangulation density. 
%
A well known result is that, to locally approximate a smooth convex $C^2$ function $f$,  the optimal shapes of triangles is dictated by the Hessian $H f$ of the function (see~\cite{shewchuk2002good}). In practice, people use $\mu(x) = |H f(x)|^\al$ for some exponent $\al > 0$ (which is related to the quality measure of the approximation), where $|\cdot|^\al$ indicates the spectral application of the exponentiation (as for matrix exp or log).
%
Figure~\eqref{fig:meshing} shows that Q-OT can be used (using formula~\eqref{eq-interpolating}) to interpolate between two sizing fields $(\mu,\nu)$, which are computed from the Hessian (with here $\al=1$) of two initial input images $(f,g)$.
%
The resulting anisotropic triangulations are defined as geodesic Delaunay triangulations for the Riemannian metric defined by the tensor field, and are computed using the method detailed in~\cite{peyre-iccv-09}
%
This interpolation could typically be used to track the evolution of the solution of some PDE. 

%%% FIG %%%
\newcommand{\MeshingImg}[2]{\includegraphics[width=.195\linewidth]{meshing/#1/input-#2}}
\begin{figure}\centering
\begin{tabular}{@{}c@{}c@{}c@{}c@{}c@{}}
\MeshingImg{2d-bump-donut}{mesh-1}&
\MeshingImg{2d-bump-donut}{mesh-3}&
\MeshingImg{2d-bump-donut}{mesh-5}&
\MeshingImg{2d-bump-donut}{mesh-7}&
\MeshingImg{2d-bump-donut}{mesh-9}\\
\MeshingImg{images}{mesh-1}&
\MeshingImg{images}{mesh-3}&
\MeshingImg{images}{mesh-5}&
\MeshingImg{images}{mesh-7}&
\MeshingImg{images}{mesh-9}\\
$t=0$ & $t=1/4$ & $t=1/2$ & $t=3/4$ & $t=1$
\end{tabular}
\begin{tabular}{@{}c@{\hspace{1mm}}c@{\hspace{8mm}}c@{\hspace{1mm}}c@{}}
\MeshingImg{images}{images-1}&
\MeshingImg{images}{mesh-1-img}&
\MeshingImg{images}{images-2}&
\MeshingImg{images}{mesh-9-img} \\
$f$ and $\mu$ & & $g$ and $\nu$ &
\end{tabular}
\caption{Two examples of interpolation between two input sizing fields $(\mu_{t=0},\mu_{t=1})=(\mu,\nu)$. 
\textbf{First row:} triangulation evolution for the sizing fields displayed on Figure~\ref{fig:intro}.
\textbf{Second row:} the input sizing fields $(\mu_{t=0},\mu_{t=1})=(\mu,\nu)$ are displayed on the third row, and are defined using the absolute valued ($\al=1$) of the Hessian of the underlying images $(f,g)$.
} \label{fig:meshing}
\end{figure}
%%% FIG %%%


