- classical versus Q-OT comparison
- sinkhorn convergence speed

- Is Figure 4 computed numerically or using the explicit formula? (I'd mention this)
- Section 5.3 paragraph 2: \tau should be \rho. Also, the parameters values in figure 8 and the text disagree (0.5 and 0.05).
- contraint on trace.

- explain computational grid size, etc.


- detail a bit more the dimension of the variables involved (dimension of u, v, etc).
- introduce also a relaxation parameter tau for the v scaling of barycenters.
- show figure with cross tranport between mixtures of 2 diracs, comparing ansiotropic vs isotropic (classical) OT. 


- SIGGRAPH requires that we cite our own Arxiv paper, which I added in the acknowledgements section.  Please triple-check that I did this right.

- Please register on the SIGGRAPH site and add the proper submission number to the paper.

- In Prop 1, do we need to mention that u,v are scalars per point?  Since some of our other variables are matrices?

- For some reason there's a tiny amount of space under each figure caption.  Is there a weird LaTeX option somewhere?

- In Figure 7, maybe show the tensor field?  [Or if the red stuff is the tensor field, it's not too visible]

- I might recommend splitting sec 5.4 into multiple paragraphs.

- Figure 9 looks much nicer than the previous version!  Can we include a few more examples of the same thing on different textures?  Right now the paper contains only one example of each application, which may look suspicious to reviewers.


===//===

- discuss criteria for termination : either some form of hilbert metric between two iterates, or a primal-dual gap (or both)
- motivate more the correcting factors A_ki in the interpolation formula, and maybe use a cleaner / more mathematically rigorous (?) formula 
- correct warning in bibtex